%% Copernicus Publications Manuscript Preparation Template for LaTeX Submissions
%% ---------------------------------
%% This template should be used for copernicus.cls
%% The class file and some style files are bundled in the Copernicus Latex Package, which can be downloaded from the different journal webpages.
%% For further assistance please contact Copernicus Publications at: production@copernicus.org
%% https://publications.copernicus.org/for_authors/manuscript_preparation.html


%% Please use the following documentclass and journal abbreviations for preprints and final revised papers.

%% 2-column papers and preprints
\documentclass[journal abbreviation, manuscript]{copernicus}



%% Journal abbreviations (please use the same for preprints and final revised papers)


% Advances in Geosciences (adgeo)
% Advances in Radio Science (ars)
% Advances in Science and Research (asr)
% Advances in Statistical Climatology, Meteorology and Oceanography (ascmo)
% Annales Geophysicae (angeo)
% Archives Animal Breeding (aab)
% Atmospheric Chemistry and Physics (acp)
% Atmospheric Measurement Techniques (amt)
% Biogeosciences (bg)
% Climate of the Past (cp)
% DEUQUA Special Publications (deuquasp)
% Drinking Water Engineering and Science (dwes)
% Earth Surface Dynamics (esurf)
% Earth System Dynamics (esd)
% Earth System Science Data (essd)
% E&G Quaternary Science Journal (egqsj)
% European Journal of Mineralogy (ejm)
% Fossil Record (fr)
% Geochronology (gchron)
% Geographica Helvetica (gh)
% Geoscience Communication (gc)
% Geoscientific Instrumentation, Methods and Data Systems (gi)
% Geoscientific Model Development (gmd)
% History of Geo- and Space Sciences (hgss)
% Hydrology and Earth System Sciences (hess)
% Journal of Bone and Joint Infection (jbji)
% Journal of Micropalaeontology (jm)
% Journal of Sensors and Sensor Systems (jsss)
% Magnetic Resonance (mr)
% Mechanical Sciences (ms)
% Natural Hazards and Earth System Sciences (nhess)
% Nonlinear Processes in Geophysics (npg)
% Ocean Science (os)
% Polarforschung - Journal of the German Society for Polar Research (polf)
% Primate Biology (pb)
% Proceedings of the International Association of Hydrological Sciences (piahs)
% Safety of Nuclear Waste Disposal (sand)
% Scientific Drilling (sd)
% SOIL (soil)
% Solid Earth (se)
% The Cryosphere (tc)
% Weather and Climate Dynamics (wcd)
% Web Ecology (we)
% Wind Energy Science (wes)


%% \usepackage commands included in the copernicus.cls:
%\usepackage[german, english]{babel}
%\usepackage{tabularx}
%\usepackage{cancel}
%\usepackage{multirow}
%\usepackage{supertabular}
%\usepackage{algorithmic}
%\usepackage{algorithm}
%\usepackage{amsthm}
%\usepackage{float}
%\usepackage{subfig}
%\usepackage{rotating}

\newcommand{\hydronntwo}{Hydronn$_{2, \text{All}}$}
\newcommand{\hydronnfourall}{Hydronn$_{4, \text{All}}$}
\newcommand{\hydronnfourir}{Hydronn$_{4, \text{IR}}$}

\begin{document}

\title{An improved near real-time precipitation retrieval for Brazil}


% \Author[affil]{given_name}{surname}

\Author[1]{Simon}{Pfreundschuh}
\Author[1]{Ingrid}{Ingemarsson}
\Author[1]{Patrick}{Eriksson}
\Author[2]{Daniel A.}{Vila}
\Author[3]{Alan J. P.}{Calheiros}

\affil[1]{Department of Space, Earth and Environment, Chalmers University of Technology, Gothenburg, Sweden}
\affil[2]{Regional office for the Americas, World Meteorological Organization, Asunción, Paraguay}
\affil[3]{Coordination of Applied Research and Technological Development, National Institute for Space Research (INPE), São José dos Campos, Brazil}

%% The [] brackets identify the author with the corresponding affiliation. 1, 2, 3, etc. should be inserted.

%% If an author is deceased, please mark the respective author name(s) with a dagger, e.g. "\Author[2,$\dag$]{Anton}{Smith}", and add a further "\affil[$\dag$]{deceased, 1 July 2019}".

%% If authors contributed equally, please mark the respective author names with an asterisk, e.g. "\Author[2,*]{Anton}{Smith}" and "\Author[3,*]{Bradley}{Miller}" and add a further affiliation: "\affil[*]{These authors contributed equally to this work.}".


\correspondence{Simon Pfreundschuh (simon.pfreundschuh@chalmers.se)}

\runningtitle{An improved nowcasting rain retrieval for Brazil}

\runningauthor{Simon Pfreundschuh}





\received{}
\pubdiscuss{} %% only important for two-stage journals
\revised{}
\accepted{}
\published{}

%% These dates will be inserted by Copernicus Publications during the typesetting process.


\firstpage{1}

\maketitle



\begin{abstract}
  Observations from geostationary satellites offer the unique ability to provide
  spatially continuous coverage at continental scales with high spatial and
  temporal resolution. Because of this, they are commonly used to complement
  ground-based measurements of precipitation, whose coverage is often more
  limited.

  We present a novel, neural-network-based, near real-time precipitation
  retrieval for Brazil based on visible and infrared (VIS/IR) observations from
  the Advanced Baseline Imager  on the Geostationary Operational
  Environmental Satellite 16. The retrieval, which employs a convolutional
  neural network to perform Bayesian retrievals of precipitation, was developed
  with the aims of (1) leveraging the full potential of latest-generation
  geostationary observations and (2) providing probabilistic precipitation
  estimates with well-calibrated uncertainties. The retrieval is trained using
  co-locations with combined radar and radiometer retrievals from the Global
  Precipitation Measurement (GPM)  Core Observatory. Its accuracy is
  assessed using one month of gauge measurements and compared to the
  precipitation retrieval that is currently in operational use at the Brazilian
  Institute for Space Research as well as two state-of-the-art global
  precipitation products: The Precipitation Estimation from Remotely Sensed
  Information using Artificial Neural Networks Cloud Classification System and
  the Integrated Multi-Satellite Retrievals for GPM (IMERG). Even in its most
  basic configuration, the accuracy of the proposed retrieval is similar to that
  of IMERG, which merges retrievals from VIS/IR and microwave observations with
  gauge measurements. In its most advanced configuration, the retrieval reduces
  the mean absolute error for hourly accumulations by $22\ \unit{\%}$ compared
  the currently operational retrieval, by $50\ \unit{\%}$ for the MSE and
  increases the correlation by $400\ \unit{\%}$. Compared to IMERG, the
  improvements correspond to  $15\ \unit{\%}$, $15\ \unit{\%}$ and
  $39\ \unit{\%}$, respectively. Furthermore, we show that the probabilistic
  retrieval is well calibrated against gauge measurements when differences in a
  priori distributions are accounted for.

  In addition to potential improvements in near real time precipitation
  estimation over Brazil, our findings highlight the potential of specialized
  data driven retrievals that are made possible through advances in
  geostationary sensor technology, the availability of high-quality reference
  measurements from the GPM mission and modern machine learning techniques.
  Furthermore, our results show the potential of probabilistic precipitation
  retrievals to better characterize the observed precipitation
  and provide more trustworthy retrieval results.
\end{abstract}


\introduction  %% \introduction[modified heading if necessary]

Timely and highly-resolved measurements of precipitation constitute an important
source of information for weather forecasting, disaster response and
hydrological modeling. These measurements can be provided by dense radar and
gauge networks but their coverage is typically limited in less populated
regions. However, even where these measurements are available, they are not
necessarily without issues. The ability of rain gauges measurements to
truthfully represent spatial precipitation statistics at larger scales is
limited by their extreme localization \citep{smith96}. Ground-based
precipitation radars are affected by beam blocking as well as measurement errors
caused by the varying altitude of the radar beam along its range
\citep{holleman07}.

Since satellite observations provide continuous spatial coverage, they are well
suited to complement the measurements from gauges and ground radars. Microwave
observations generally provide the most direct space-borne measurements of
precipitation because of their sensitivity to emission and scattering from
precipitating hydrometeors. Unfortunately, due to their comparably low spatial
resolution, these sensors are currently employed only on polar orbiting
platforms. Since this limits the width of the satellite swath, a large
constellation of sensors on different platforms is required to achieve low
revisit times. This approach is pursued by the Global Precipitation Measurement
(GPM, \citeauthor{hou14}, \citeyear{hou14}). Nonetheless, the revisit times for
the passive microwave sensors of the GPM constellation still exceed 2 hours ins
the tropics.

Visible and infrared (VIS/IR) observations from the latest generation of
geostationary satellites \citep{schmit05} provide spatial resolutions between
0.5 and $2\ \unit{km}$ at the sub-satellite point and a temporal resolution of
up to 10 minutes for full disk observations. The disadvantage of these
observations for measuring precipitation is that they are mostly sensitive to
the top of clouds, which are only indirectly related to the precipitation near
the surface. Their unrivaled spatial and temporal resolution nonetheless makes
them a valuable source of information for satellite-based precipitation
estimates.

The operational use of geostationary VIS/IR observations for precipitation
retrievals dates back more than 40 years \citep{scofield77} and a large number
of different algorithms have been developed over the years \citep{arkin87,
  adler88, vicente98, sorooshian00, kuligowski02, roderick03, hong04,
  kuligowski16}. Due to the aforementioned indirect relationship between
observations and precipitation, nearly all of these methods are based on
empirical relationships derived from satellite observations co-located with
reference data derived from more direct measurement techniques such as
ground-based radar. Moreover, operational retrievals often require corrections
to improve the accuracy of their estimates. The Self-Calibrating Real-Time GOES
Rainfall Algorithm for Short-Term Rainfall Estimates (SCamPR,
\citeauthor{kuligowski16} \citeyear{kuligowski16}), for example, is dynamically
calibrated using the latest available microwave precipitation estimates.
Similarly, the PERSIANN CSS real time dataset is in the process of being
superseded by the PERSIANN PDIR algorithm, which extends the PERSIANN CCS
algorithm with a regional correction.

%Differences between algorithms arise with respect to the assumed form of the
%empirical relationship and the input observations: The Autoestimator
%\citep{vicente98} and Hydroestimator \citep{roderick03} are based on an assumed
%power-law relationship between $10.7\ \unit{\mu m}$ IR observations and surface
%precipitation. The PERSIANN retrieval \citep{hsu97} uses a neural network to
%classify cloud pixels and map them to precipitation rates. The PERSIANN cloud
%classification system (CCS) \citep{hong04} extends this approach by using a
%neural network to classify cloud pixels and then applying different empirical
%power-law relationships relating cloud top temperature to precipitation rates.
%Operational algorithms also typically employ a range of corrections to improve
%retrieval results across meteorological and climatic regimes.

Another example is the HYDRO precipitation retrieval that is currently in
operational use at the National Institute for Space Resarch (INPE) in Brazil,
which is based on the Hydroestimator algorithm \citep{scofield03}. It employs an
empirical relationship between the $10.7\ \unit{\mu m}$ IR channel and
precipitation rates with additional corrections. To adapt it for application
over South America yet another correction was derived by \citet{siqueira19},
which improved the accuracy of precipitation accumulations but not that of
instantaneous precipitation rates.

A common shortcoming of all retrieval algorithms discussed above is that they
neglect retrieval uncertainties. The retrieval of precipitation rates from
VIS/IR observations constitutes an inverse problem that is strongly
underconstrained. This is true even for microwave based retrievals and likely
exacerbated by the less direct information content in the VIS/IR observations.
The ill-posed character of the retrieval problem leads to significant retrieval
uncertainties. Providing probabilistic estimates that quantify these
uncertainties would help the characterization of precipitation estimates and
thus increase their usefulness.

This study presents Hydronn, a novel real-time precipitation retrieval that uses
VIS/IR observations from the GOES 16 Advanced Baseline Imager (ABI,
\citeauthor{schmit18}, \citeyear{schmit18}) to retrieve precipitation over
Brazil. It was designed with two aims: (1) To leverage the full potential of
observations from the latest generation of geostationary sensors and (2) to
develop a Bayesian precipitation retrieval algorithm that can provide well
calibrated uncertainty estimates. The algorithm is trained using a large
database of co-locations between observations from the ABI and the combined
radar-radiometer retrievals from the GPM Core Observatory \citep{grecu16} . The
accuracy of the retrieval is evaluated using a month of gauge measurements. To
assess its performance with respect to existing retrievals it is compared to the
currently operational HYDRO algorithm as well as two other commonly used
precipitation products: The hourly Precipitation Estimation from Remotely Sensed
Information using Artificial Neural Networks Cloud Classification System
(PERSIANN CCS, \citeauthor{hong04}, \citeyear{hong04}) , which is based on IR
observations only, and the Integrated Multi-Satellite Retrievals for GPM (IMERG,
\citeauthor{huffman20} \citeyear{huffman20}), which combines observations from
microwave, geostationary sensors and rain gauges.

%The remainder of the article is structured as follows:
%Section~\ref{sec:data_and_methods} introduces the data product that are provide
%the benchmark estimates in this study, followed by a description of the Hydronn
%retrieval algorithm. Section~\ref{sec:results} first evaluates the different
%configurations of Hydronn on separate test data and then assesses their accuracy
%against the benchmark algorithms on the gauge data. Setion~\ref{sec:discussion}
%discusses the results in the more general context of the development of
%precipitation retrievals while Sec.~\ref{sec:conclusions} summarizes the main
%results of this study.


\section{Data}

This section presents the rain gauge data, which serve as reference measurements
for this study, as well as the baseline algorithms to which Hydronn will be
compared. This is followed by a description of the Hydronn retrieval algorithm.

\subsection{Rain gauge data}

The rain gauge measurements that are used in this study were compiled by the
National Institute of Meteorology of Brazil and consists of hourly gauge
measurement covering the time range May 2000 until May 2020. From this data
December of 2020 will be used for the evaluation of Hydronn. Data from 2018 and
2019 is used to derive correction factors for the calibration of the hourly
precipitation estimates produced by Hydronn, as will be described in
Sec.~\ref{sec:correction} below.

From all available gauge stations only those with a data availability exceeding
$90\ \unit{\%}$ during December 2020 were selected. Their geographical
distribution is displayed together with the mean precipitation in
Fig.~\ref{fig:gauges}. While the gauge density is fairly high on the
south-eastern coast of Brazil, it decreases markedly towards the Northwest.
Climatologically, the highest precipitation occurs in the Amazonas and
surrounding regions in the Northwest of the country, however this is not well
represented in the gauge measurements due to their sparsity in this region. More
evident are the high precipitation amounts on the south-western coast of the
country extending towards the Northwest, which manifest the South Atlantic
Convergence Zone (SACZ, \citeauthor{satyamarty98}, \citeyear{satyamarty98}).
Very low precipitation rates are observed in the Northeast, which is influenced
by large scale subsistence patterns \citep{siqueira19}.

\begin{figure}
  \centering
  \includegraphics[width=0.9\textwidth]{figs/fig01}
  \caption{
    Overview of the rain gauge data from December 2020 used to validate the
    retrievals. Panel (a) displays their spatial distribution by means of the
    number of gauges falling into each hexagon. Hexagon-free areas are not
    covered by any gauges. Panel (b) shows the corresponding mean precipitation.
    }
  \label{fig:gauges}
\end{figure}
    

\subsection{HYDRO}

HYDRO is the currently operational near real-time precipitation retrieval at the
Center for Weather Forecast and Climate Studies/National Institute for Space
Research (CPTEC/INPE). It is based on the Hydroestimator \citep{scofield03} and
thus uses a combination of empirical power-law relationships between
$10.7\ \unit{\mu m}$ IR brightness temperatures and surface precipitation with
correction factors taking into account model-derived moisture and wind
parameters as well as cloud structure. The current version of the retrieval is
described in \citet{siqueira19}, which also introduces regional correction
factors based on a climatology of surface precipitation rates derived from radar
measurements of the Tropical Rainfall Measurement Mission (TRMM,
\citeauthor{simpson96}, \citeyear{simpson96}) and GPM. For this study we use the
corrected version of HYDRO proposed in \citet{siqueira19} with a regional
correction for all of Brazil (referred to as HYDROBR in \citet{siqueira19}).


\subsection{PERSIANN CSS}

PERSIANN CCS \citep{hong04} uses $10.7\ \unit{\mu m}$ IR observations from
geostationary satellites to retrieve precipitation. Input images are first
segmented using increasing temperature thresholds in order to identify pixels
that correspond to convective activity. These pixels are consecutively assumed
to be precipitating and classified using a neural network based algorithm.
Quantitative precipitation estimates at pixel level are derived from this
classification by applying a class-specific power law relationship that relates
the $10.7\ \unit{\mu m}$ brightness temperatures to precipitation.

The dataset that is used for the evaluation against Hydronn are hourly
precipitation rates that are distributed in near real time through the PERSIANN
data portal \citep{persiann_data}. Although the global CCS dataset is currently
being replaced by the updated PERSIANN Dynamic Infrared-Rain rate (PDIR-Now,
\citeauthor{nguyen20}, \citeyear{nguyen20}) we were not able to use it for this
study due to the first third of the evaluation period missing from the online
archive.


\subsection{GPM IMERG}

GPM IMERG \citep{huffman20} combines retrievals from passive microwave and IR
observations as well as rain gauge measurements to produce global, half-hourly
measurements of precipitation. Due to its reliance on a wealth of measurement
sources as well as the sophistication of the retrieval pipeline, the product can
be considered one of the most robust satellite-based precipitation products
\citep{pradhan22}.

Three different configurations of IMERG products are available: IMERG-Early and
IMERG-Late are based solely on satellite observations and available with
latencies of 4 and 14 hours, respectively. IMERG-Final is adjusted using global
gauge measurements but available first after 3.5 months. Although Hydronn has
been designed to target near real-time applications and is thus more similar to
IMERG-Early, we use IMERG-Final for our comparison as it constitutes the most
elaborate precipitation estimates that are currently available and can thus be
considered the state of the art of space-borne quantitative precipitation
estimation.

\section{Method}

This section describes the implementation of Hydronn, the proposed near
real-time precipitation retrieval algorithm for Brazil. It is based on a
convolutional neural network (CNN), which is used to predict the a posteriori
distribution of instantaneous precipitation. Following this, it is discussed how
the probabilistic precipitation estimates can be combined to hourly
accumulations and an a priori adjustment is proposed to account for differences
between the training data and the gauge measurements that are used to evaluate
the retrieval.


\subsection{Training data}

The training data for the Hydronn retrieval is generated from co-locations of
input observations from the GOES-16 ABI \citep{schmit18} and retrieved surface
precipitation from the GPM combined product \citep{grecu16}. GOES ABI
observations were extracted at their native resolutions and combined with the
GPM combined surface precipitation by remapping them to the $2\ \unit{km}$
resolution of the ABI's IR channels using a nearest-neighbor criterion.
Co-locations were extracted for the time range 2018-01-01 until 2020-01-01 and
2021-01-01 until 2021-09-01. Observations from the first, 11th and 21st day of
every month of 2020 are used as test data set to establish the nominal
performance of different retrieval configurations.

The results presented in \citet{pfreundschuh18} show the correspondence between
the training data of neural network retrievals and the a priori distribution of
explicitly Bayesian retrieval schemes. This perspective emphasizes the
importance of training data distribution for the retrieval results and their
interpretation. The distribution of precipitation rates in the training dataset
is displayed in Fig.~\ref{fig:training_data}. The detection threshold for
precipitation of the GPM radar between $0.2$ and $0.4\ \unit{mm \ h^{-1}}$ is
clearly visible in the distributions. In addition to this, a weak seasonal cycle
is apparent, which mainly impacts the likelihood of moderate precipitation. The
gauge measurements exhibit a stronger effect of the seasonal cycle especially
for strong rain. It should be noted here, however, that the precipitation
estimates in the training data correspond to instantaneous precipitation
estimates while the gauge measurements are integrated over the time of an hour.
Differences between the seasonal cycles of the datasets may therefore be caused
by changes in the temporal evolution of precipitation events. An approach to
reconcile the differences between the distributions of training and validation
will be proposed in Sec.~\ref{sec:correction} below.


\begin{figure}
  \centering
  \includegraphics[width=1.0\textwidth]{figs/fig02}
  \caption{
    Distribution of reference precipitation rates. Panel (a) shows the
    seasonal PDFs of precipitation rates in the training data. Panel (b)
    shows the PDFs of precipitation rates measured by the gauges
    over the time period covered by the training data. Grey lines in the
    background trace the PDFs of the precipitation rates in the training data
    shown in Panel (a).
    }
  \label{fig:training_data}
\end{figure}


\subsection{Retrieval configurations}

The Hydronn retrieval has been implemented in three different configurations in
order to assess how the choice of input observations and output resolution
affects its performance. The most basic retrieval configuration is the
\hydronnfourir{} retrieval, which only uses brightness temperatures from the
GOES 16 $10.3\ \unit{\mu m}$ channel as input. Due to their sensitivity to cloud
top temperatures, longwave IR window channels are also used by HYDRO as well as
the PERSIANN CCS retrieval. The availability of similar channels on a long range
historical geostationary sensors makes them suitable for the generation of
climate data records. The reliance on a single thermal IR channel has the
additional advantage that the information content of the retrieval input is
independent of the availability of sunlight and thus constant throughout the
day. The second retrieval configuration, denoted as \hydronnfourall{}, uses all
available GOES channels at a resolution of $4\ \unit{km}$. It uses the same
neural network model as the \hydronnfourir{} configuration adapted to the larger
number of input channels. The third configuration, \hydronntwo{}, aims to
exploit the full potential of GOES observations for precipitation retrievals.
The input includes observations from all GOES channels at their native
resolution and precipitation is retrieved at a resolution of $2\ \unit{km}$ at
nadir. The characteristics of the three configurations are summarized in
Tab.~\ref{tab:configurations}.

\begin{table}[hbpt!]
  \centering
  \caption{Hydronn retrieval configurations}
  \label{tab:configurations}
  \begin{tabular}{l|rrr}
  Name & Input bands & Input resolution & Output resolution \\
  \hline
  \hydronnfourir{} & $13$ & $4\ \unit{km}$ & $4 \ \unit{km}$ \\
  \hydronnfourall{} & $1, \ldots, 16$ & $4\ \unit{km}$ & $4 \ \unit{km}$ \\
  \hydronntwo{} & 1, \ldots, 16 & $0.5, 1$ and $2 \unit{km}$ & $2 \ \unit{km}$
  \end{tabular}
\end{table}


\subsection{Neural network model}

All Hydronn retrievals are based on a common convolutional neural network (CNN)
architecture, which is illustrated in Fig.~\ref{fig:architecture}. A preliminary
study found CNNs to yield significantly more accurate results than
fully-connected neural networks that use only a single pixel as input
\citep{ingemarsson21}. The fully-convolutional networks are constructed using
what we refer to as Xception blocks, which are based on the Xception
architecture proposed by \citet{chollet17}. These blocks are combined in an
asymmetric encoder-decoder structure with 5 stages. Each downsampling stage
consists of one downsampling Xception block followed by $N=4$ standard Xception
blocks.

Since the \hydronntwo{} retrieval ingests observations at their native
resolution, this architecture contains two additional downsampling blocks that
are omitted for the \hydronnfourir{} and \hydronnfourall{} retrievals. The
number of internal features for all architectures was set to $n_f = 128$, which
is probably low compared to other neural network architectures. This was mostly
motivated by hardware limitations. Since it was found to be sufficient to
achieve good retrieval performance, we did not investigate the impact of this
decision further.

\begin{figure}[hbpt!]
  \centering
  \includegraphics[width=0.8\textwidth]{figs/fig03}
  \caption{
    The neural network architecture used by the \hydronntwo{} retrieval.
  }
  \label{fig:architecture}
\end{figure}

\subsection{Probabilistic precipitation estimates} 

A defining characteristic of Hydronn is that precipitation is retrieved using a
Bayesian framework. This means that, instead of predicting a single precipitation
value, it provides an estimate of the full a posteriori distribution of the
Bayesian retrieval problem. Although \citet{pfreundschuh18} proposed to use
quantile regression neural networks (QRNNs) to perform Bayesian remote sensing
retrievals with neural networks, a different approach is taken here. Following
the work by \citet{sonderby20}, the range of precipitation values is discretized
and the probability of the observed precipitation falling into each bin is
predicted. By normalizing the predicted probabilities, a binned approximation of
the probability density function (PDF) of the Bayesian a posterior distribution
can be obtained. We found this approach to be equivalent to QRNNs in retrieval
accuracy. However, calculating the distribution of the sum of two temporally
independent predictions is easier on the binned PDF than on the predicted
quantiles, which why the former approach was chosen for the implementation of
Hydronn.

This approach, which we refer to for simplicity as density regression neural
network (DRNN), can be easily implemented by treating the retrieval as a
classification problem over a discretized range of precipitation values and
using a cross-entropy loss to train the network. During inference, the logits
predicted by the network are transformed into a probability density by applying
a softmax activation and normalizing the bin probabilities.

Hydronn predicts the a posteriori distribution over 128 logarithmically spaced
bins covering the range from $10^{-3}$ to $10^3\ \unit{mm\ h^{-1}}$. The
reference precipitation of pixels without rain was set to a log-uniform random
value between $10^{-3}$ and $10^{-2}\ \unit{mm\ h^{-1}}$. While not strictly
necessary for our approach, this has the advantage of breaking the degeneracy of
low quantiles of the posterior distribution, which makes it easier to verify the
calibration of the probabilistic predictions on the validation data.

Since storing the full posterior distribution for all pixels is of little use
for operational processing, only a reduced number of relevant statistics are
retained in the retrieval output. Those are the posterior mean as well as a
sample and 14 quantiles of the posterior distribution. In addition to providing
a direct measure of retrieval uncertainty, the quantiles can be used to
reconstruct a piece-wise linear approximation of the posterior cumulative
distribution function (CDF). The predicted posterior CDF can then be used, for
example, to detect the exceedance of certain precipitation thresholds. Compared
to training a separate classifier to perform this task, this approach has the
advantage that the precipitation threshold can be chosen during inference.

\subsection{Calculation of hourly accumulations}
\label{sec:accumulation}

The precipitation estimates produced by Hydronn correspond to instantaneous
precipitaiton rates. Since GOES 16 imagery is available every 10 minutes, a
method is required to accumulate the posterior distributions of the
instantaneous precipitation rates to hourly accumulations, which can then be
compared to the gauge measurements. While this is not an issue when only the
posterior mean is predicted, it is unclear how the retrieval uncertainties
should be aggregated in time. In lack of a formal way to resolve this, we have
implemented two heuristics for calculating probabilistic estimates of hourly
accumulations from instantaneous measurements.

The first heuristic is to simply average the predicted posterior distributions.
For the case of multiple identical observations, this preserves the retrieval
uncertainties and thus corresponds to the assumption of strong dependence of the
retrieval errors for consecutive observations. The second approach is to assume
temporal independence of the retrieval uncertainty. For identical, consecutive
observations this will generally cause the retrieval uncertainty to decay. Given
the binned probability densities of two independent random variables, the PDF of
their sum can be approximated by calculating weighted histograms of the outer
sum of the bin centroids weighted by the product of the bin probabilities.

For the evaluation of the Hydronn retrieval, we calculate PDFs of hourly
accumulations using both approaches. Two types of accumulations are produced for
each Hydronn configuration: One corresponding to the assumption of dependent
retrieval errors, which will be identified with the qualifier '(dep.)', as well as
one corresponding to the assumption of independent retrieval errors, which will
be identified with the qualifier '(indep.)'. Since the assumptions only affect
the predicted retrieval uncertainties and not the predicted mean values, such a
distinction is not required when point estimates of precipitation are
considered.

\subsection{Correcting for a priori data}
\label{sec:correction}

According to Bayes theorem, the posterior distribution of  retrieved
precipitation $p(x|\mathbf{y})$ for  given input observations $\mathbf{y}$ is
proportional to the product of the probability of observing $\mathbf{y}$ for a
given precipitation rate $x$ and the a priori probability of $x$:
\begin{align}
  p(x|\mathbf{y}) \propto p(\mathbf{y}|x) p(x)
\end{align}
 One difficulty with machine learning based retrievals is that the a priori
 distribution cannot be chosen freely but is dictated by the training data
 distribution. Fig.~\ref{fig:training_data} indicates that there are
 inconsistencies between the training data and gauge measurements. For example,
 the retrieval will learn from the training data that the probability of
 precipitation values between $10^{-1}$ and $10^{-2}\ \unit{mm\ h^{-1}}$ is
 effectively zero.

This raises the question whether it is possible to correct for the effect of the
a priori assumptions encoded in the training data of the retrieval. To explore
this, we propose the following method to correct the probabilistic predictions.
Let $p_\text{Gauges}(x)$ denote the PDF of precipitation as measured by the
available gauges shown in Fig~\ref{fig:training_data} (b). Moreover let
\begin{align}
  r(x) &= \frac{p_\text{Gauges}(x)}{p(x)}
 \end{align}
denote the ratio of the PDFs of the gauge a priori distribution and the a priori
distribution of precipitation as defined by the training data. Assuming that
$p_\text{Gauges}(x) = 0$ wherever $p(x) = 0$ and that the conditional
distribution $p(\mathbf{y}|x)$ of the observations remains unchanged, a
corrected posterior distribution can be obtained by point-wise multiplying the
likelihood ratio $r$ with the posterior distribution predicted by Hydronn:
\begin{align}
  p_\text{Corrected}(x|y) \propto p(y|x) r(x) p(x),
\end{align}

The a priori distribution of retrieved hourly accumulations is not necessarily
the distribution of instantaneous rain rates displayed in
Fig.~\ref{fig:training_data}, but depends on how these accumulations are
calculated. This leads to a different a priori distributions for each of the two
methods used to accumulate the precipitation (see Sec.~\ref{sec:accumulation}).
For the assumptions of temporally dependent retrieval uncertainties, the a
priori is identical to that of the instantaneous rain rates. For the assumption
of temporally independent retrieval uncertainties, however, the a priori
corresponds to the distribution of the mean of 6 consecutive draws from this
distribution.

An issue that we encountered here is the limited numerical resolution of the
gauge measurements of $0.2\ \unit{mm\ h^{-1}}$, which causes problems when the
calibration of probabilistic predictions is evaluated against the gauge
measurements. To counteract this, we have added numerical noise to the gauge
measurements: Uniform random values from the range $[-0.1 , 0.1]$ are added to
all non-zero measurements, while zero values are replaced by log-randomly
distributed values from the range $[10^{-3}, 5 \cdot 10^{-3}]$. This slightly
increases the mean of the precipitation by about $2\ \unit{\%}$, but this is
likely negligible compared to other uncertainties that affect precipitation
retrievals.

\begin{figure}[hbpt!]
  \centering
  \includegraphics[width=0.8\textwidth]{figs/fig04}
  \caption{
    A priori distributions of hourly accumulations and derived correction
    factors $r$. Panel (a) displays the a priori distributions of hourly
    precipitation accumulations derived assuming strong temporal dependence of
    measurements (green) and complete independence (blue). The gray, dashed line
    shows the PDF of the gauge measurements. Panel (b) displays the
    corresponding correction factors for the two assumptions calculated as the
    ratio between the respective PDFs and the PDF of the gauge measurements.
  }
  \label{fig:correction_factors}
\end{figure}

The a priori distributions and corresponding derived correction factors are
displayed in Fig.~\ref{fig:correction_factors}. It is apparent that the
assumptions of temporally dependent uncertainties yields better agreement with
the gauge data than the assumption of temporally independent uncertainties. The
resulting correction factors are thus closer to the $y=1$ line for the
dependence assumption. We found that it was necessary to truncate the correction
factors corresponding to the independence assumption at $r=10^3$ because larger
values would amplify numerical noise leading to the rare occurrence of
unrealistically high precipitation values, which would distort the retrieval
results.


\section{Results}

This section presents the evaluation of the Hydronn retrievals, which is split
into three parts. The first part analyzes the nominal performance of the three
Hydronn configurations on the held-out test data. The second part compares the
retrieved hourly accumulations to the gauge measurements and the reference
precipitation algorithms. Finally, the third part presents a case study of
a heavy precipitation event that occurred during the validation period.

\subsection{Evaluation of Hydronn configurations}

To obtain an unperturbed assessment of the relative performances of the three
Hydronn configurations, we evaluate the performance on the test data, which was
derived from the same source (albeit during a different time period) and is
therefore guaranteed to have similar statistics as the training data. Fig.
\ref{fig:evaluation_scatter} displays PDFs of retrieved precipitation
conditioned on the value of the reference precipitation. Due to the limited
information content of the VIS/IR observations there are significant
uncertainties in all results. These lead to significant wet biases for lightly
raining pixels and dry biases for strong precipitation.

Nonetheless, slight improvements between the three Hydronn configurations are
discernible. While the \hydronnfourir{} retrieval exhibits the weakest
relationship between reference and retrieved precipitation, the \hydronnfourall{}
configuration yields slightly more accurate results. This can be seen in the
sharpening of the conditional PDFs for precipitation rates occurring between $2$
and $20\ \unit{mm\ h^{-1}}$ as well as an increase in the slope of the
conditional mean retrieved precipitation for rain rates exceeding
$2\ \unit{mm\ h^{-1}}$. Clearer improvements in retrieval accuracy are observed
for the \hydronntwo{} configuration, which yields a slightly sharper 
distribution and an increased slope in the conditional mean of the retrieved
precipitation for precipitation rates larger than $0.5\ \unit{mm\ h^{-1}}$.

\begin{figure}[hbpt]
  \centering \includegraphics[width=1.0\textwidth]{figs/fig05}
  \caption{PDFs of retrieved precipitation conditioned on the reference
    precipitation. The purple line shows the mean retrieved precipitation
    conditioned on the value of the reference precipitation.}
  \label{fig:evaluation_scatter}
\end{figure}

For a more quantitative analysis, Tab.~\ref{tab:evaluation_metrics} summarizes
the retrieval performance using a range of accuracy metrics. The metrics
considered here are the bias, mean absolute error (MAE), mean squared error (MSE),
the mean of the continuous ranked probability score (MCRPS) and the correlation
coefficient. Given a predicted cumulative distribution function $F$ and a
reference value $x$, the continuous ranked probability score (CRPS) is defined
as
\begin{align}
  \text{CRPS}(F, x) = \int_{-\infty}^\infty (F(x') - \mathrm{I}_{x' > x})^2\ dx',
\end{align}
where $\mathrm{I}_{x' > x}$ is the indicator function taking the value $1$ where
$x' > x$ and $0$ otherwise.

The 14 quantiles that are produced as retrieval output are used to calculate the
CRPS instead of the full predicted PDF, which ensures that the evaluation is
representative of the actual retrieval output. In contrast to the other metrics
considered in Tab.~\ref{tab:evaluation_metrics}, the CRPS takes into account not
only the accuracy of the predicted posterior mean but both sharpness and
calibration of the probabilistic precipitation estimates \citep{gneiting07}.

Overall, the results from Tab.~\ref{tab:evaluation_metrics} confirm the
tendencies observed in Fig~\ref{fig:evaluation_scatter}. The retrieval accuracy
increases as the information content in the input observations is increased. In
absolute terms, the largest improvements are achieved when the inputs are
extended from a single channel to all channels of the ABI. However, further
improvements can be achieved by ingesting all observations at their native
resolutions and retrieving precipitation at $2\ \unit{km}$ resolution. It should
be noted that the test dataset contains observations from all times of the day,
so these improvements are not constrained by the availability of daylight.

\begin{table}
  \caption{Accuracy metrics for the three Hydronn configurations evaluated on
    test data. The value corresponding to the highest accuracy in each column
  is marked using bold font.}
  \label{tab:evaluation_metrics}

\begin{tabular}{l|rrrrr}
  Algorithm & Bias [$\unit{mm\ h^{-1}}$] & MAE [$\unit{mm\ h^{-1}}$] & MSE [$\unit{(mm\ h^{-1})^{2}}$]& CRPS & Correlation  \\
  \hline
  \hydronnfourir{} & 0.0009 & 0.1988 & 1.3356 & 0.1807 & 0.4343 \\
  \hydronnfourall{} & 0.0006 & 0.1626 & 1.1848 & 0.0865 & 0.5296 \\
  \hydronntwo{} & \bf{-0.0001} & \bf{0.1508} & \bf{1.0827} & \bf{0.0757} & \bf{0.5844} \\
\end{tabular} 
\end{table}


\subsection{Validation against rain gauge data}

In contrast to Hydronn, all of the reference algorithms considered here neglect
the probabilistic nature of the retrieval and provide only a single
precipitation estimate. The evaluation of Hydronn against the reference
algorithms is therefore performed in two steps. Firstly, the accuracy of the
deterministic quantitative precipitation estimates is evaluated against the
gauge measurements. Secondly,  the probabilistic estimates produced by
Hydronn and their potential to improve the characterization of the
observed precipitation are assessed.

\subsubsection{Quantitative precipitation estimates}

Accuracy metrics of all retrievals evaluated against the gauge measurements for
hourly, daily and monthly precipitation means are provided in
Tab.~\ref{tab:metrics}. In terms of correlations for hourly means, HYDRO yields
the worst performance with a correlation of 0.134, followed by PERSIANN CCS with
a correlation of 0.26. IMERG and \hydronnfourir{} achieve similar accuracy for
hourly estimates with a correlation around 0.4. The \hydronnfourall{} and
\hydronntwo{} retrievals further improve the accuracy with correlations of 0.5
and 0.545, respectively. As the integration time increases, the accuracy of all
retrievals improves. For daily means, the ranking of the retrieval algorithms
remains the same as for hourly means. This is also the case for monthly means
with the exception that the accuracy of IMERG increases to the level of the best
Hydronn configuration. A likely explanation for this is the calibration that is
applied to the IMERG Final product, which matches it to monthly gauge
measurements. 

\begin{table}[hbpt!]
  \caption{Accuracy metrics for the retrieved mean precipitation compared to gauge measurements
    at different time scales. The best values in each column are marked using bold font.}
  \label{tab:metrics}
  \begin{tabular}{l|r|rrr|rrr|rrr}
        & &
        \multicolumn{3}{c|}{MAE [$\unit{mm\ h^{-1}}$]} &
        \multicolumn{3}{c|}{MSE [$(\unit{mm\ h^{-1}})^2$]} &
        \multicolumn{3}{c}{Correlation} \\
        \multicolumn{1}{c|}{Retrieval} &
        \multicolumn{1}{c|}{Bias [$\unit{mm\ h^{-1}}$]} &
        Hourly & Daily & Monthly &
        Hourly & Daily & Monthly &
        Hourly & Daily & Monthly \\
        \hline
        HYDRO & -0.030 &
        0.308 & 0.208  & 0.104 &
        2.964 & 0.212  & 0.019 &
        0.134 & 0.421 & 0.629 \\
        %
        PERSIANN CCS & 0.088 &
        0.382 & 0.274 & 0.144 &
        3.417 & 0.293 & 0.04 &        
        0.26  & 0.415 & 0.55 \\
        %
        IMERG & 0.015 &
        0.282 & 0.191 & 0.077 &
        2.262 & 0.178 & $\mathbf{0.013}$ &
        0.389 & 0.574 & 0.741 \\


        \hydronnfourir{} & -0.023 &
        0.277 & 0.191 & 0.096 &       
        2.091 &  0.163 &  0.017 &
        0.412 & 0.573 & 0.650 \\

        \hydronnfourall{} & $\mathbf{0.002}$ &
        0.247 & 0.168 & 0.081 &
        1.874 & 0.135 & 0.014 &
        0.502 & 0.662 & 0.731 \\

        \hydronntwo{}   &  0.011 &
        $\mathbf{0.239}$ & $\mathbf{0.162}$ & $\mathbf{0.075}$ &
        $\mathbf{1.760}$ & $\mathbf{0.128}$ & $\mathbf{0.013}$ &
        $\mathbf{0.545}$ & $\mathbf{0.685}$ & $\mathbf{0.756}$ \\
  \end{tabular}
\end{table}

A graphical analysis of the accuracy of the retrieved daily accumulations is
provided in Fig~\ref{fig:daily_accumulations}. In this representation the large
uncertainties that are present in all retrievals are evident. Nonetheless, the
results confirm the general findings from the analysis above. The two
conventional VIS/IR retrievals, HYDRO and PERSIAN CCS, yield the least accurate
results. In particular, both retrievals show a tendency to miss or strongly
underestimate accumulations below $50\ \unit{mm\ d^{-1}}$. This tendency is
decreased in the IMERG results for accumulations $> 10\ \unit{mm\ h^{-1}}$ but
still evident for weaker precipitation. Overall, the Hydronn retrievals achieve
higher accuracy for both low and high precipitations, which increases with the
information content of the input. Nonetheless, systematic underestimation of
strong rain rates affects all Hydronn retrievals.

\begin{figure}[hbpt!]
  \centering
  \includegraphics[width=1.0\textwidth]{figs/fig06}
  \caption{
    Scatter plots of retrieved daily accumulations against gauge measurements
    for the reference retrievals and the three Hydronn configurations.
    Frequencies have been normalized column-wise to improve the visibility of
    high reference precipitation.
  }
  \label{fig:daily_accumulations}
\end{figure}

The spatial distribution of the biases of the monthly mean precipitation is
displayed in Fig.~\ref{fig:mean_precipitation}. The strongest biases are
observed in the PERSIANN CCS results, which strongly overestimate precipitation
in central and northern Brazil. HYDRO and \hydronnfourir{}, as well as to a
lesser extent PERSIANN CCS, \hydronnfourall{} and \hydronntwo{}, exhibit a
systematic dry bias in southern Brazil. Overall, the biases of IMERG are
smallest in magnitude and exhibit the least extent of spatial correlation.
However, the differences between IMERG and the best Hydronn configuration,
\hydronntwo{}, are small.

\begin{figure}[hbpt!]
  \centering
  \includegraphics[width=1.0\textwidth]{figs/fig07}
  \caption{
    Mean precipitation during December 2020. Shading in the background of
    each panel shows the spatial distribution of the mean precipitation
    of the corresponding retrieval. Colored hexagons show the spatial
    distributions of the retrieval biases with respect to the gauge
    measurements.
    }
  \label{fig:mean_precipitation}
\end{figure}


Finally, we consider the derived daily cycles of precipitation, which are
displayed in Fig.~\ref{fig:daily_cycle}. From the reference retrievals, both
IMERG and HYDRO yield relatively good agreement with the gauge measurements.
IMERG is slightly closer to the gauge measurements during morning and early
afternoon but overestimates precipitation in the afternoon and evening. HYDRO
slightly underestimates precipitation during the first half of the day but its
afternoon peak, despite being close in magnitude to that of the gauge
measurements, is delayed by about three hours. PERSIANN CCS shows good agreement
with the gauge measurements in the first half of the day but strongly
overestimates the afternoon peak. All Hydronn configurations yield good
agreement with gauge measurements. The \hydronntwo{} configurations slightly
overestimates precipitation before 10 am, while \hydronnfourir{} underestimates
the afternoon peak.

\begin{figure}[hbpt]
  \centering
  \includegraphics[width=1.0\textwidth]{figs/fig08}
  \caption{
    Measured and retrieved daily cycles of precipitation. Panel (a) displays the
    daily cycles retrieved by the three reference retrievals (solid lines) and
    the gauge measurements (dashed line) for reference. Panel (b) displays the
    corresponding diurnal cycles for the three Hydronn configurations (solid
    lines).
  }
  \label{fig:daily_cycle}
\end{figure}

\subsubsection{Probabilistic estimates}

We now proceed to evaluate the probabilistic precipitation estimates that are
produced by Hydronn. As explained in Sec.~\ref{sec:accumulation}, two
probabilistic estimates of the hourly precipitation rates were produced. The
first one, (dep.), assumes strong dependence of retrieval errors for consecutive
observations, while the second one, (indep.), assumes independent errors. In
addition to that, for each of these predictions a corrected distribution has
been calculated using the correction factors described in
Sec.~\ref{sec:correction}. This yields four probabilistic predictions for each
Hydronn configuration. These predictions will be referred to with the
configuration name and the qualifiers (dep.), (dep., corr.) for the predictions
derived assuming dependent uncertaintes with and without correction,
respectively, and (indep.), (indep. corr) for the corresponding predictions
derived using the independence assumption.

We first consider the distribution of precipitation rates as measured by gauges
and the retrieval algorithms, which is shown in
Fig.~\ref{fig:rain_rate_distributions}. All reference retrievals overestimate
the frequency of low and moderate rain rates while underestimating the frequency
of high rain rates. For each Hydronn configuration, the distributions of the
mean as well as a random sample from the retrieval posterior are displayed. The
results for the three Hydronn configurations are almost identical to each other.
The distribution of the posterior mean exhibits similar characteristics as the
retrieved values of the reference retrievals. Although slightly closer to the
distribution of gauge measurements, the uncorrected distribution of samples of
the posterior distribution obtained assuming temporally independent errors still
does not match the gauge distribution very well. This is improved by the a
priori correction but an overestimation of moderate precipitation remains.
Sampling from the posterior distributions corresponding to the assumption of
temporally dependent uncertainties yields fairly good agreement with the
distribution of gauge measurements. It is improved further by the a priori
correction, which mostly improves the agreement at low and moderate rain rates.

 The deviations between the distribution of retrieved posterior mean values and
 the reference distribution can thus be understood as an effect of the
 uncertainties in the retrieval results. These uncertainties lead to a lack of
 very high precipitation values in the distribution of retrieved means of the
 retrieval posterior. When instead of the mean samples from the posterior
 distribution are considered, and the differences between the a priori and gauge
 measurement distributions are taken into account, the extreme values of the
 distribution are correctly reproduced.

\begin{figure}
  \centering
  \includegraphics[width=1.0\textwidth]{figs/fig09}
  \caption{
    Distributions of measured and retrieved rain rates. The dark grey, filled
    curve in the background of each panel shows the PDF of the gauge
    measurements. Colored lines drawn on top show the corresponding PDFs of the
    retrieved precipitation. The distributions for the reference algorithms are
    shown in Panel (a). Panel (b), (c) and (d) show the distribution of mean
    values and random samples from the retrieval posterior.
  }
  \label{fig:rain_rate_distributions}
\end{figure}

In addition to sampling from the posterior distribution, the retrieved quantiles
can be used to derive confidence intervals for the predicted precipitation.
Their reliability is assessed in Fig.~\ref{fig:calibration}, which displays the
calibration curves of the confidence intervals. While the assumption of
dependent retrieval errors leads to an overestimation of the uncertainties,
assuming independent errors leads to underestimation. However, for both
assumptions this is corrected by the a priori correction, albeit some
underestimation remains for the (indep.) results. The results presented in
Fig.~\ref{fig:calibration} use the modified gauge measurements described in
Sec.~\ref{sec:correction} to which small random noise has been added to non-zero
measurements and zeros were replaced with small random values. This was required
because quantiles are ill-defined when the CDF of a quantity is discontinuous.
So while the results above show that the predicted uncertainties from the
Hydronn retrieval are well calibrated, they would not be when compared against
the raw gauge measurements. However, since the modifications to the measured
precipitation are well within their uncertainty this  still demonstrates
the meaningfulness of the predicted confidence intervals.

\begin{figure}
  \centering
  \includegraphics[width=1.0\textwidth]{figs/fig10}
  \caption{
    Calibration of the confidence intervals derived using the quantiles
    of the retrieval posterior distribution predicted by each Hydronn
    configuration. Grey, dashed line in the background shows the expected
    results for perfectly calibrated results.
    }
  \label{fig:calibration}
\end{figure}


The retrieved quantiles can also be used derive probabilities of an observed
pixel exceeding certain precipitation thresholds. This has been used to derive
probabilities of the hourly precipitation exceeding $5$ and
$20\ \unit{mm\ h^{-1}}$, which correspond roughly to the $99$ and
$99.9\ \unit{\%}$ of the CDF of precipitation rates. The ability of the
retrievals to detect high-impact precipitation events is assessed using
precision-recall (PR) curves in Fig.~\ref{fig:pr_curves}. For the
non-probabilistic retrievals the curves were generated using the predicted
precipitation and classifying all pixel above a varying threshold as exceeding
the sought-after precipitation rate. The corresponding curves for the Hydronn
retrievals were obtained by varying the probability threshold above which a
pixel is classified as an high-impact event.

For the detection of events exceeding $5\ \unit{mm\ h^{-1}}$, HYDRO again
exhibits the least skill, followed by PERSIANN CCS. Compared to these two, IMERG
performs clearly better. The detection skill of \hydronnfourir{} is close to
that of IMERG, while the two other configurations yield significantly better
detection performance than IMERG. Interestingly, the assumption used to
accumulate the uncertainties as well as the a priori correction do not have
a significant effect on the detection skill. The reason for this is likely that
the variation of the probability threshold for the generation of the PR curves
has a calibrating effect on the retrieval results. For events exceeding $20\ \unit{mm\ h^{-1}}$,
all retrievals yield worse detection accuracy. Also here HYDRO exhibits the
least skill, followed by PERSIANN CCS and IMERG, which yield very similar
results. All Hydronn configurations outperform the reference retrievals.

\begin{figure}
  \centering
  \includegraphics[width=1.0\textwidth]{figs/fig11}
  \caption{
    Precision-recall  curves for the detection of precipitation events with
    precipitation rates larger than $5\ \unit{mm\ h^{-1}}$ (first row) and
    $20\ \unit{mm\ h^{-1}}$ (second row). Columns show the results for the
    reference retrievals as well as for each of the Hydronn configurations.
    }
  \label{fig:pr_curves}
\end{figure}

Since the Hydronn retrievals can be used to derive a probability of an
observation exceeding a given precipitation threshold, a relevant question is
how accurate these probabilities are. The calibration of the detection
probabilities of events exceeding $5\ \unit{mm\ h^{-1}}$ is displayed in
Fig.~\ref{fig:classification_calibration}. While the predictions derived
assuming temporally dependent uncertainties are fairly well calibrated, larger
deviations are observed for the probabilities derived assuming independent
uncertainties. Moreover, while the correction marginally improves the
calibration of the results for the dependent uncertainties, it degrades that of
the probabilities derived assuming independent uncertainties. While the results
are qualitatively similar across all Hydronn configurations, the detection
probabilities increase in magnitude with the information content of the
retrieval inputs, indicating that the sharpness of the probabilistic predictions
increases. Nonetheless, even for an event with a threshold of only
$5\ \unit{mm\ h^{-1}}$ probabilities do not exceed $60\ \unit{\%}$, which
highlights the limitations of the VIS/IR based retrievals for the detection of
extreme precipitation.

\begin{figure}
  \centering
  \includegraphics[width=1.0\textwidth]{figs/fig12}
  \caption{
    Calibration of the probabilistic precipitation event detection. 
    }
  \label{fig:classification_calibration}
\end{figure}

In addition to the PR curves, we have calculated commonly used detection metrics
for all retrievals in order to simplify the comparison to other studies. These
metrics include the probability of detection (POD, the fraction of events that
were correctly detected), the false alarm rate (FAR, the fraction of all
detections that are wrong) and the critical success index (CSI) for the
detection of events with more than $20\ \unit{mm \ h^{-1}}$. The results are
displayed in Tab.~\ref{tab:detection}. For the calculation of POD and FAR for
the reference retrievals, a retrieved precipitation rate exceeding
$20\ \unit{mm\ h^{-1}}$ is counted as a detection. For the Hydronn retrievals
the probability threshold has been tuned to yield a FAR close to that of IMERG.
As was to be expected from the PR curves in Fig.~\ref{fig:pr_curves}, HYDRO
exhibits the least detection skill and does not detect any of the actual
occurrences of strong precipitation but still produces false positives. Compared
to this, IMERG does significantly better but suffers from a very low POD.
PERSIANN performs better than IMERG in terms of POD but also has a higher FAR.
The Hydronn retrievals achieve higher POD at lower FAR for all configurations.
The POD of extreme precipitation increases markedly as more channels and higher
resolution are incorporated. Similar results are observed for the CSI.

\begin{table}
  \caption{Error metrics for the detection of precipitation events with rain rates exceeding
    $20\ \unit{mm\ h^{-1}}$.}
  \label{tab:detection}
\begin{tabular}{ll|rrrr}
Retrieval     & &   POD &   FAR &   CSI\\
\hline
PERSIANN CCS  & & 0.006 & 0.987 &  0.004 \\
PERSIANN CCS  & & 0.039 & 0.926 &  0.026 \\
HYDRO         & & 0.000 & 1.000 &  0.000 \\
IMERG         & & 0.018 & 0.867 &  0.016 \\
\hline
\multirow{4}{*}{Hydronn$_{4, \text{IR}}$} & (dep.)          & 0.085 & 0.803 &  0.063 \\
& (dep., corr.)   & 0.121 & 0.811 &  0.080 \\
& (indep.)        & 0.079 & 0.806 &  0.059 \\
& (indep., corr.) & 0.076 & 0.798 &  0.058 \\
\hline
\multirow{4}{*}{Hydronn$_{4, \text{All}}$} &  (dep.)         & 0.109 & 0.800 &  0.076 \\
&  (dep., corr.)  & 0.133 & 0.808 &  0.085 \\
&  (indep.)       & 0.167 & 0.801 &  0.100 \\
&  (indep., corr.)& 0.145 & 0.800 &  0.092 \\
\hline
\multirow{4}{*}{Hydronn$_{2, \text{All}}$} &  (dep.)         & 0.242 & 0.799 &  0.123 \\
&  (dep., corr.)  & 0.270 & 0.804 &  0.128 \\
&  (indep.)       & 0.267 & 0.799 &  0.129 \\
&  (indep., corr.)& 0.242 & 0.801 &  0.123 \\
\end{tabular}
\end{table}


\subsection{Case study}

As final part of this evaluation, a case of heavy precipitation in the city of
Duque de Caxias in the State of Rio de Janeiro is considered, which occurred
between the 22nd and 24th of December 2020 and lead to floodings
\citep{flooding}. About $250\ \unit{mm}$ of accumulated precipitation were
measured by the rain gauge in Xer\'em over the period of two days. An overview
of the spatial distribution of retrieved accumulated precipitation over the two
days is provided in Fig.~\ref{fig:accumulations_case}. Fairly good agreement is
observed between IMERG and the \hydronnfourall{} and \hydronntwo{}
configurations. \hydronnfourir{} yields accumulations of slightly lower
magnitude and is spatially less well defined than the two other Hydronn
retrievals. The accumulations from PERSIANN CCS are lowest in terms of
magnitude, whereas those from HYDRO are highest. However, neither of the two
agrees well with the spatial distribution retrieved by IMERG or the Hydronn
retrievals. Both PERSIANN CCS and HYDRO exhibit systematic dry biases in the
region around Duque de Caxias compared to the gauge measurements, which is
present to a lesser degree also in the results of \hydronnfourir{}. This
dry bias is less pronounced in the results of IMERG, \hydronnfourall{} and
\hydronntwo{}.

\begin{figure}[!hbpt]
  \centering
  \includegraphics[width=1.0\textwidth]{figs/fig13}
  \caption{
    Retrieved precipitation for an extreme precipitation event in the city of
    Duque de Caxias in the state of Rio de Janeiro. Shading in background shows
    the accumulated precipitation between 2020-12-22 00:00 to 2020-12-24 00:00.
    Colored markers show the relative bias of the retrieved accumulated
    precipitation compared to gauges. The green star marks the location of
    Xer\'em, where flooding occurred.
  }
  \label{fig:accumulations_case}
\end{figure}

The rain rates at the gauge station in Xer\'em (location marked by green star in
Fig.~\ref{fig:accumulations_case}) are displayed in
Fig.~\ref{fig:precipitation_case}. The plots show the hourly precipitation rates
retrieved by the reference retrievals as well as the mean and posterior
distribution for all Hydronn retrievals. Only results obtained with the
assumption of dependent retrieval uncertainties and the a priori correction
applied are shown. The precipitation measured at the rain gauge far exceeds the
precipitation measured by any of the reference retrievals or the retrieved mean
of the Hydronn retrievals. The Hydronn retrievals predict elevated uncertainties
for the period during which the strongest precipitation is observed. However,  the
precipitation peaks still exceed the 99th percentile. Two factors may explain
that more than the expected $1\ \unit{\%}$ of gauge measurements lie outside the
predicted uncertainty range. Firstly, the observations considered here are not
randomly sampled but correspond to an event that is known to be extreme.
Secondly, as stated in the article in \citet{flooding}, heavy precipitation
events are common in this region. This may indicate that  regional factors
may act to intensify the precipitation, which is unlikely to be captured in the
training data of the retrieval.

Nonetheless, an encouraging results is that the predicted value of the 99th
percentile increases with the information content in the retrieval input. This
indicates that the neural network can leverages the additional information to
produce sharper uncertainty estimates.

\begin{figure}[!hbpt]
  \centering
  \includegraphics[width=1.0\textwidth]{figs/fig14}
  \caption{
    Retrieved precipitation for an extreme precipitation even that occurred between
    2020-12-22 and 2020-12-24 in the cite of Duque de Caxias in the
    state of Rio de Janeiro. Grey, dashed lines show the precipitation by the
    gauge station in Xer\'em. Solid lines show the retrieved mean precipitation
    for each retrieval algorithm. The shading shows filled contours of the
    posterior CDF at values $[0.01, 0.1, 0.2, \ldots, 0.8, 0.9, 0.99]$.
    }
  \label{fig:precipitation_case}
\end{figure}

\section{Discussion}

The study presented Hydronn, a neural-network-based precipitation retrieval for
Brazil, which has been trained using combined radar and radiometer measurements
from the GPM Core Observatory satellite. The retrieval compares favorably
against the currently operational precipitation retrieval, HYDRO, as well as the
PERSIANN CCS product. In its best configuration the Hydronn retrieval yields
retrieval accuracy superior to that of the IMERG Final product across most
considered metrics.

\subsection{Information content of VIS/IR observations}

The three tested retrieval configurations use input observation of increasing
information content. The \hydronnfourir{} configuration only uses a single IR
channel at a resolution of $4\ \unit{km}$ while \hydronnfourall{} uses all
available bands. The best performing retrieval, \hydronntwo{}, combines the
observations from all channels of the GOES ABI at their native resolutions.
Clear increases in retrieval performance are observed when all ABI bands are
incorporated into the retrieval as well as when all channels are ingested at
their native resolutions. This demonstrates the ability of the
neural-network-based retrieval to learn complex relationships even from input
observations with low information content. Moreover, the fact that HYDRO and
\hydronnfourir{} use the same observations as retrieval input further highlights
the need for advanced statistical retrieval techniques to fully exploit the
potential of current geostationary satellite observation. This is in good
agreement with the results found by \citet{sadeghi19}.

\subsection{Probabilistic precipitation retrievals}

A novel aspect of the proposed precipitation retrievals is their ability to
provide probabilistic precipitation estimates. In this study we have
demonstrated multiple ways in how this may improve the utility of the retrieval
results:

\begin{enumerate}
  \item The results in Fig.~\ref{fig:rain_rate_distributions} show that samples
    from the retrieval posterior reproduce the gauge-measured distribution of
    rain rates fairly accurately. The deviations of the distribution of the
    posterior mean from the gauge measurements can thus be understood as a
    consequence by the statistical properties of this estimator instead of a
    retrieval deficiency. The random samples may be useful for applications that
    are sensitive to heavy precipitation rates, such as run off modeling or
    climatological studies. As an example, Fig.~\ref{fig:quantile_scatter} shows
    scatter plots of the 99th percentile of the distribution of gauge-measured
    and retrieved precipitation during December 2020 for all gauge station. Also
    here the Hydronn retrievals yield the best estimates. For this evaluation,
    HYDRO and PERSIANN CCS yield similar accuracy as IMERG despite IMERG having
    higher accuracy for all other metrics considered in this study. This is
    likely because both HYDRO and PERSIANN CCS were both developed to correctly
    represent heavy precipitation, which harms their accuracy in terms of other
    statistics. By explicitly resolving the probabilistic nature of the
    precipitation retrieval, HYDRONN can provide both climatologically accurate
    accumulations (see Tab.~\ref{tab:metrics}) and correct representation of
    heavy precipitation.

    It should be noted, however, that these random samples do not take into
    account spatial correlations. To what extent this may negatively impact
    applications of the retrieval results remains to be investigated in a
    follow-up study.

\item The retrieved quantiles allow the derivation of confidence intervals to
  quantify retrieval uncertainty. By correcting for the difference in a priori
  distributions as well as the degeneracy of quantiles due to discontinuities in
  the CDF of gauge measurements, we were able to show that the retrieval
  uncertainties are well calibrated even against gauge measurements
  (Fig.~\ref{fig:calibration}). Due to the large uncertainties that are inherent
  to precipitation retrieval from VIS/IR observations
  (Fig.~\ref{fig:evaluation_scatter}, \ref{fig:daily_accumulations}),
  quantifying them increases the trustworthiness of the predictions.
\item We have shown that the retrieved quantiles can be used to detect
  heavy precipitation events (Fig.~\ref{fig:pr_curves},
  Tab.~\ref{tab:metrics}). Here all Hydronn retrievals perform better than IMERG
  although they are based on observations with a significantly lower information
  content. This clearly shows the benefits of quantifying the retrieval
  uncertainties. Moreover, we were able to show that the probabilistic detection
  of these events is fairly well calibrated
  (Fig.\ref{fig:classification_calibration}).

  \begin{figure}[!hbpt]
    \centering
    \includegraphics[width=0.8\textwidth]{figs/fig15}
    \caption{
      The 99th quantile of the distribution of gauge measurements of each gauge
      station plotted against the 99th quantile of the corresponding
      distribution of retrieved precipitation.
    }
    \label{fig:quantile_scatter}
  \end{figure}

  These results, however, also show the limitation of VIS/IR retrievals since
  even for events exceeding $5\ \unit{mm\ h^{-1}}$ the maximum detection
  confidence is $60\ \unit{\%}$. This certainly puts the suitability of any of
  the considered retrievals to detect heavy precipitation events into question
  and calls for new approaches that combine observations from different
  observational sources.
\end{enumerate}

Finally, we have also investigated how uncertainties from instantaneous
precipitation retrievals can be propagated to the full hour. The two approaches
that we have tested correspond to assuming temporally independent and temporally
dependent retrieval uncertainties. Our results indicate
(Fig.~\ref{fig:rain_rate_distributions}, Fig.~\ref{fig:calibration}) that the
assumption of dependent uncertainties overestimates the actual retrieval
uncertainty, whereas assuming independent uncertainties underestimates actual
uncertainties. It is interesting to note that the way the uncertainties are
accumulated does not affect the detection of extreme events
(Fig.~\ref{fig:pr_curves}), which indicates that the probabilities could also
be re-calibrated a posteriori.

\subsection{Utility of a priori corrections}

We have proposed a method to correct for the distribution of precipitation rates
in the training data. The corrections have improved the agreement between the
distribution of retrieved precipitation rates as well as the calibration of the
uncertainty intervals (Fig.~\ref{fig:rain_rate_distributions},
Fig.~\ref{fig:calibration}). Although for the assumed independent uncertainties
the calibration was improved, the distribution of precipitation rates did
exhibit slight deviations from the distribution of the gauge measurements. We
suspect that the reason for the correction working worse in the latter case is
that the a corresponding a priori assumption deviates stronger from the
distribution of the gauge measurements (Fig.~\ref{fig:correction_factors}). This
led to much higher correction factors, which were truncated to avoid numerical
issues.

The clearest effect of the a priori correction was observed when the predicted
confidence intervals were evaluated against gauge data
(Fig.~\ref{fig:calibration}). This allowed us to show that the Hydronn
retrievals can provide well-calibrated uncertainty estimates for their
predictions when the differences between the a priori distributions of the
training data and the gauge measurements are taken into account. Nonetheless,
the correction did not affect the detection of strong precipitation. We suspect
the reason for this to be that the correction mostly affects small precipitation
rates due to their high occurrence in the training and validation data as well
as the re-calibrating effect of the varying probability threshold in the
generation of the PR curves.

The correction relies on the assumptions that the conditional distribution of
the observations vector $p(\mathbf{y}|x)$ given the rain rate $x$ remains
constant. It thus can only correct for differences in the measurement
characteristics between the rain gauge data, which is used to evaluate the
retrieval, and the GPM data, which was used to derive the training data. It can
not, however, correct for differences between the training and evaluation data
that involves changes in the observed processes, which would change
$p(\mathbf{y}|x)$. We argue that this is not an issue for this study since the
evaluation and training data are overlapping geographically.

\conclusions

Hydronn, the presented neural-network-based precipitation retrieval, improves
real time precipitation estimates over Brazil. Its performance is superior to
both the currently operational algorithm as well as the much more complex global
IMERG Final product, which combines observations from both VIS/IR and passive
microwave sensors as well as global gauge measurements.

Our results demonstrate the potential of designing region specific retrieval
algorithms, which exploit the full potential of locally available satellite
observations. This is made possible by the availability of accurate surface
precipitation retrievals from the GPM CO satellite, which were used to derived
the training data for the retrieval. Since this data is available globally
between $-60$ and $60\ \unit{^\circ N}$, the approach can potentially be applied
to most other regions around the world.

Although our evaluation focused on Brazil, many of the results presented here
should be of interest for precipitation retrievals from geostationary satellites
in general. In addition to providing further evidence of the potential of deep
neural networks to improve quantitative precipitation estimates, we show how a
probabilistic regression approach can be used to perform VIS/IR precipitation
retrievals using a Bayesian framework and that the probabilistic predictions help
to better characterize the retrieval results and identify heavy precipitation.

Finally, the fact that our relatively simple retrieval outperforms state of the
art precipitation products despite being solely based on VIS/IR observations
shows the potential of algorithmic innovation for quantitative precipitation
estimation. The ability of the neural network retrieval to leverage information
from all channels of the ABI at their native resolutions, shows the strength of
the end-to-end approach to retrieval design. This suggests that further
improvements for precipitation retrievals should be achievable by expanding the
retrieval input to incorporate additional spectral as well as temporal
information.




%% It is strongly recommended to make use of these sections in case data sets and/or software code have been part of your research the article is based on.

\codeavailability{
  The code to generate the retrieval training data, train the retrieval models, run the retrievals and analyze
  the results is available in public repositor \citep{hydronn22}.
} %% use this section when having only software code available


%\dataavailability{TEXT} %% use this section when having only data sets available


%\codedataavailability{TEXT} %% use this section when having data sets and software code available


%\sampleavailability{TEXT} %% use this section when having geoscientific samples available


%\videosupplement{TEXT} %% use this section when having video supplements available


\appendix
\section{}    %% Appendix A

\subsection{}     %% Appendix A1, A2, etc.


\noappendix       %% use this to mark the end of the appendix section. Otherwise the figures might be numbered incorrectly (e.g. 10 instead of 1).

%% Regarding figures and tables in appendices, the following two options are possible depending on your general handling of figures and tables in the manuscript environment:

%% Option 1: If you sorted all figures and tables into the sections of the text, please also sort the appendix figures and appendix tables into the respective appendix sections.
%% They will be correctly named automatically.

%% Option 2: If you put all figures after the reference list, please insert appendix tables and figures after the normal tables and figures.
%% To rename them correctly to A1, A2, etc., please add the following commands in front of them:

\appendixfigures  %% needs to be added in front of appendix figures

\appendixtables   %% needs to be added in front of appendix tables


\authorcontribution{II, PE and SP designed the study. SP and II developed the retrieval and analyzed the retrieval results. SP prepared the manuscript. AC and DV provided the gauge measurements, HYDRO retrieval results and valuable feedback. } %% this section is mandatory

\competinginterests{No competing iterests are present.} %% this section is mandatory even if you declare that no competing interests are present

\begin{acknowledgements}
  We would like to acknowledge the Brazilian National Institute of Meteorology
  for the provision of the gauge measurements.
\end{acknowledgements}



%% REFERENCES



\bibliographystyle{copernicus}
\bibliography{references}
%% Since the Copernicus LaTeX package includes the BibTeX style file copernicus.bst,
%% authors experienced with BibTeX only have to include the following two lines:
%%
%% \bibliographystyle{copernicus}
%% \bibliography{example.bib}
%%
%% URLs and DOIs can be entered in your BibTeX file as:
%%
%% URL = {http://www.xyz.org/~jones/idx_g.htm}
%% DOI = {10.5194/xyz}


%% LITERATURE CITATIONS
%%
%% command                        & example result
%% \citet{jones90}|               & Jones et al. (1990)
%% \citep{jones90}|               & (Jones et al., 1990)
%% \citep{jones90,jones93}|       & (Jones et al., 1990, 1993)
%% \citep[p.~32]{jones90}|        & (Jones et al., 1990, p.~32)
%% \citep[e.g.,][]{jones90}|      & (e.g., Jones et al., 1990)
%% \citep[e.g.,][p.~32]{jones90}| & (e.g., Jones et al., 1990, p.~32)
%% \citeauthor{jones90}|          & Jones et al.
%% \citeyear{jones90}|            & 1990



%% FIGURES

%% When figures and tables are placed at the end of the MS (article in one-column style), please add \clearpage
%% between bibliography and first table and/or figure as well as between each table and/or figure.

% The figure files should be labelled correctly with Arabic numerals (e.g. fig01.jpg, fig02.png).


%% ONE-COLUMN FIGURES

%%f
%\begin{figure}[t]
%\includegraphics[width=8.3cm]{FILE NAME}
%\caption{TEXT}
%\end{figure}
%
%%% TWO-COLUMN FIGURES
%
%%f
%\begin{figure*}[t]
%\includegraphics[width=12cm]{FILE NAME}
%\caption{TEXT}
%\end{figure*}
%
%
%%% TABLES
%%%
%%% The different columns must be seperated with a & command and should
%%% end with \\ to identify the column brake.
%
%%% ONE-COLUMN TABLE
%
%%t
%\begin{table}[t]
%\caption{TEXT}
%\begin{tabular}{column = lcr}
%\tophline
%
%\middlehline
%
%\bottomhline
%\end{tabular}
%\belowtable{} % Table Footnotes
%\end{table}
%
%%% TWO-COLUMN TABLE
%
%%t
%\begin{table*}[t]
%\caption{TEXT}
%\begin{tabular}{column = lcr}
%\tophline
%
%\middlehline
%
%\bottomhline
%\end{tabular}
%\belowtable{} % Table Footnotes
%\end{table*}
%
%%% LANDSCAPE TABLE
%
%%t
%\begin{sidewaystable*}[t]
%\caption{TEXT}
%\begin{tabular}{column = lcr}
%\tophline
%
%\middlehline
%
%\bottomhline
%\end{tabular}
%\belowtable{} % Table Footnotes
%\end{sidewaystable*}
%
%
%%% MATHEMATICAL EXPRESSIONS
%
%%% All papers typeset by Copernicus Publications follow the math typesetting regulations
%%% given by the IUPAC Green Book (IUPAC: Quantities, Units and Symbols in Physical Chemistry,
%%% 2nd Edn., Blackwell Science, available at: http://old.iupac.org/publications/books/gbook/green_book_2ed.pdf, 1993).
%%%
%%% Physical quantities/variables are typeset in italic font (t for time, T for Temperature)
%%% Indices which are not defined are typeset in italic font (x, y, z, a, b, c)
%%% Items/objects which are defined are typeset in roman font (Car A, Car B)
%%% Descriptions/specifications which are defined by itself are typeset in roman font (abs, rel, ref, tot, net, ice)
%%% Abbreviations from 2 letters are typeset in roman font (RH, LAI)
%%% Vectors are identified in bold italic font using \vec{x}
%%% Matrices are identified in bold roman font
%%% Multiplication signs are typeset using the LaTeX commands \times (for vector products, grids, and exponential notations) or \cdot
%%% The character * should not be applied as mutliplication sign
%
%
%%% EQUATIONS
%
%%% Single-row equation
%
%\begin{equation}
%
%\end{equation}
%
%%% Multiline equation
%
%\begin{align}
%& 3 + 5 = 8\\
%& 3 + 5 = 8\\
%& 3 + 5 = 8
%\end{align}
%
%
%%% MATRICES
%
%\begin{matrix}
%x & y & z\\
%x & y & z\\
%x & y & z\\
%\end{matrix}
%
%
%%% ALGORITHM
%
%\begin{algorithm}
%\caption{...}
%\label{a1}
%\begin{algorithmic}
%...
%\end{algorithmic}
%\end{algorithm}
%
%
%%% CHEMICAL FORMULAS AND REACTIONS
%
%%% For formulas embedded in the text, please use \chem{}
%
%%% The reaction environment creates labels including the letter R, i.e. (R1), (R2), etc.
%
%\begin{reaction}
%%% \rightarrow should be used for normal (one-way) chemical reactions
%%% \rightleftharpoons should be used for equilibria
%%% \leftrightarrow should be used for resonance structures
%\end{reaction}
%
%
%%% PHYSICAL UNITS
%%%
%%% Please use \unit{} and apply the exponential notation


\end{document}

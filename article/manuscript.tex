%% Copernicus Publications Manuscript Preparation Template for LaTeX Submissions
%% ---------------------------------
%% This template should be used for copernicus.cls
%% The class file and some style files are bundled in the Copernicus Latex Package, which can be downloaded from the different journal webpages.
%% For further assistance please contact Copernicus Publications at: production@copernicus.org
%% https://publications.copernicus.org/for_authors/manuscript_preparation.html


%% Please use the following documentclass and journal abbreviations for preprints and final revised papers.

%% 2-column papers and preprints
\documentclass[journal abbreviation, manuscript]{copernicus}



%% Journal abbreviations (please use the same for preprints and final revised papers)


% Advances in Geosciences (adgeo)
% Advances in Radio Science (ars)
% Advances in Science and Research (asr)
% Advances in Statistical Climatology, Meteorology and Oceanography (ascmo)
% Annales Geophysicae (angeo)
% Archives Animal Breeding (aab)
% Atmospheric Chemistry and Physics (acp)
% Atmospheric Measurement Techniques (amt)
% Biogeosciences (bg)
% Climate of the Past (cp)
% DEUQUA Special Publications (deuquasp)
% Drinking Water Engineering and Science (dwes)
% Earth Surface Dynamics (esurf)
% Earth System Dynamics (esd)
% Earth System Science Data (essd)
% E&G Quaternary Science Journal (egqsj)
% European Journal of Mineralogy (ejm)
% Fossil Record (fr)
% Geochronology (gchron)
% Geographica Helvetica (gh)
% Geoscience Communication (gc)
% Geoscientific Instrumentation, Methods and Data Systems (gi)
% Geoscientific Model Development (gmd)
% History of Geo- and Space Sciences (hgss)
% Hydrology and Earth System Sciences (hess)
% Journal of Bone and Joint Infection (jbji)
% Journal of Micropalaeontology (jm)
% Journal of Sensors and Sensor Systems (jsss)
% Magnetic Resonance (mr)
% Mechanical Sciences (ms)
% Natural Hazards and Earth System Sciences (nhess)
% Nonlinear Processes in Geophysics (npg)
% Ocean Science (os)
% Polarforschung - Journal of the German Society for Polar Research (polf)
% Primate Biology (pb)
% Proceedings of the International Association of Hydrological Sciences (piahs)
% Safety of Nuclear Waste Disposal (sand)
% Scientific Drilling (sd)
% SOIL (soil)
% Solid Earth (se)
% The Cryosphere (tc)
% Weather and Climate Dynamics (wcd)
% Web Ecology (we)
% Wind Energy Science (wes)


%% \usepackage commands included in the copernicus.cls:
%\usepackage[german, english]{babel}
%\usepackage{tabularx}
%\usepackage{cancel}
%\usepackage{multirow}
%\usepackage{supertabular}
%\usepackage{algorithmic}
%\usepackage{algorithm}
%\usepackage{amsthm}
%\usepackage{float}
%\usepackage{subfig}
%\usepackage{rotating}

\newcommand{\hydronntwo}{Hydronn$_{2}$}
\newcommand{\hydronnfourall}{Hydronn$_{4}$}
\newcommand{\hydronnfourir}{Hydronn$_{4, \text{IR}}$}

\begin{document}

\title{An improved nowcasting precipitation retrieval for Brazil}


% \Author[affil]{given_name}{surname}

\Author[1]{Simon}{Pfreundschuh}
\Author[1]{Ingrid}{Ingemarsson}
\Author[1]{Patrick}{Eriksson}
\Author[1]{Daniel Alejandor}{Vila}
\Author[1]{Alan}{Calheiros}

\affil[1]{Department of Space, Earth and Environment, Chalmers University of Technology, 41296 Gothenburg, Sweden}

%% The [] brackets identify the author with the corresponding affiliation. 1, 2, 3, etc. should be inserted.

%% If an author is deceased, please mark the respective author name(s) with a dagger, e.g. "\Author[2,$\dag$]{Anton}{Smith}", and add a further "\affil[$\dag$]{deceased, 1 July 2019}".

%% If authors contributed equally, please mark the respective author names with an asterisk, e.g. "\Author[2,*]{Anton}{Smith}" and "\Author[3,*]{Bradley}{Miller}" and add a further affiliation: "\affil[*]{These authors contributed equally to this work.}".


\correspondence{Simon Pfreundschuh (simon.pfreundschuh@chalmers.se)}

\runningtitle{An improved nowcasting rain retrieval for Brazil}

\runningauthor{Simon Pfreundschuh}





\received{}
\pubdiscuss{} %% only important for two-stage journals
\revised{}
\accepted{}
\published{}

%% These dates will be inserted by Copernicus Publications during the typesetting process.


\firstpage{1}

\maketitle



\begin{abstract}
\end{abstract}


\introduction  %% \introduction[modified heading if necessary]


\section{Data and methods}

This section introduces the rain gauge measurements that are used as reference
data in this study together with the retrieval algorithms that the Hydronn
retrieval will be evaluated against. Following this, the implementation of the
hydronn retrieval is presented.

\subsection{Rain gauge data}

The rain gauge measurement that are used in this study were compiled by the
Institute for Space Research (INPE) and consists of hourly gauge measurements
covering the time range May 2000 until May 2020. From this data December of
2020 will be used for the evaluation of the Hydronn retrieval while the data
from 2018 and 2019 will be used to derive a priori corrections.

The geographical distribution and the mean precipitation during December 2012 of
gauges is displayed in Fig.~\ref{fig:gauges}. While the gauge density is fairly
high on the south-western coast of Brazil it decreases markedly towards the
North-West. The highest precipitation is observed on the lower-central west
coast of the country extending into the north west. Very low precipitation rates
are observed in the north east.

\begin{figure}
  \centering
  \includegraphics[width=1.0\textwidth]{../plots/gauges}
  \caption{
    Rain gauge data used in this study. Panel (a) displays the spatial
    distribution of gauges by means of the number of gauges falling into
    each hexagon. Hexagon-free areas are not covered by gauges. Panel (b)
    shows the mean precipitation rate in each gauge.
    }
  \label{fig:gauges}
\end{figure}
    

\subsection{HYDRO}

HYDRO is the currently operational real-time precipitation retrieval at the
Center for Weather Forecast and Climate Studies/National Institute for Space
Research (CPTEC/INPE). It is based on the auto-estimator \citep{vicente98}, a
semi-empirical retrieval technique using a power-law relationship between
$10.8\ \unit{\mu m}$ IR brightness temperatures and correction factors taking
into account moisture parameters and cloud structure. The current version of the
retrieval is described in \citet{siqueira19}, which also introduces regional
correction factors derived from a climatology of estimated surface precipitation
from the radars of the Tropical Rainfall Measurement Mission (TRMM,
\citeauthor{simpson96}, \citeyear{simpson96}) and GPM. For this study we use the
corrected version of HYDRO proposed in \citet{siqueira19} with the regional
corrections applied. As is was found to yield the most accurate results.

\subsection{PERSIANN CSS}

The PERSIANN Cloud Classification System (CCS, \citeauthor{hong04},
\citeyear{hong04}) is another precipitation retrieval algorithm that uses IR
observations from geostationary satellites to retrieve precipitation. It is
included in this study as a reference dataset due to it being available globally
at a 1-hour resolution. The algorithm is a fully empirical algorithm based on
self-organizing maps, an early form of neural networks, that are used to
classify cloud patches and associate them to power-law relationships, which are
used to estimate the precipitation at each pixel. Although the global CCS
dataset is currently being replaced by the updated PERSIANN Dynamic
Infrared-Rain rate (PDIR-Now, \citeauthor{nguyen20}, \citeyear{nguyen20}) we
decided not to use it due to the first half of the evaluation period missing
from the online archive.


\subsection{GPM IMERG}

The Integrated Multi-satellite Retrievals for GPM (IMERG,
\citeauthor{huffman15}, \citeyear{huffman15}) combines retrievals from PMW
and IR observations as global rain gauge measurements to produce global,
hourly measurements of precipitation. Due to its reliance on a wealth of
measurement sources and the sophistication of the retrieval pipeline, the
product can be considered one of the most accurate precipitation hourly
precipitation products. 

\subsection{Hydronn}

Hydronn is a neural network based real time precipitation retrieval algorithm
that uses IR and VIS observation from the GOES-16 geostrationary satellite. It
has been developed specifically to retrieve precipitation over Brazil.

\subsubsection{Training data}

The training data for the Hydronn retrieval consists of co-locations of input
observations from the GOES-16 ABI \citep{schmit18} and retrieved surface
precipitation from the GPM combined product \citep{grecu16}. GOES ABI
observations were extracted at their native resolutions and combined with the
surface precipitation retrieved from the GPM combined product by mapping it to
the $2\ \unit{km}$ resolution of the ABI's IR channels. Co-locations were
extracted for the time span 2018-01-01 until 2021-09-01. Observations from the
year 2020 were excluded from the training and are only used in the evaluation of
the hydronn configurations.

The distribution of precipitation rates in the training dataset is displayed in
Fig.~\ref{fig:training_data}. The detection threshold for precipitation of the
GPM radar between $0.2$ and $0.4\ \unit{mm h^{-1}$ is clearly visible in the
  distributions. In addition to this, there is a moderate seasonal cycle, which
  mainly impacts the likelihood of moderate precipitation. The gauge
  measurements exhibit a stronger effect of the seasonal cycle especially for
  strong rain. Despite those differences, it cannot be concluded that the
  training and gauge data are inconsistent, since the differences may be due to
  the satellite measurements corresponding to instantaneous measurements,
  whereas the gauge measurements are accumulated over the span of an hour.


\begin{figure}
  \centering
  \includegraphics[width=1.0\textwidth]{../plots/training_data_statistics}
  \caption{
    Distribution of reference precipitation rates. Panel (a) shows the
    seasonal PDFs of precipitation rates in the training data. Panel (b)
    shows the PDFs of precipitation rates measured by the gauges
    over the time period covered by the training data. Grey lines in the
    background trace the PDFs of the precipitation rates in the training data
    shown in Panel (a).
    }
  \label{fig:training_data}
\end{figure}


\subsubsection{Retrieval configurations}

The Hydronn retrieval has been implemented in three different configurations in
order to assess how the choice of input observations and retrieval resolution
affects the retrieval. The most basic retrieval configuration is the
\hydronnfourir retrieval, which only uses the $10.7\ \unit{\mu\ m}$ channel.
This configuration has the advantage of using only a single infrared channel,
that is available from all currently operational geostationary
satellites and most historical sensors. The reliance on IR observations alone has
furthermore the advantage the information content of the input observation is
independent of the availability of sunlight. Because of it's sensitivity to cloud
top height this channel has a long history of being used in precipitation retrievals
and is also the sole channel used by the HYDRO and PERSIANN CCS algorithms.

The next configuration named \hydronnfourall uses all available GOES channels at
a resolution of $4\ \unit{km}$ at zenith. Except for relying  more satellite
observations as input data, this configuration  is identical to the \hydronnfourir
configuration.

The final configuration aims to exploit the full potential of GOES observations
for precipitation retrievals. The input includes observations from all GOES
channels at their native resolution while precipitation is retrieved at a
resolution of $2\ \unit{km}$ at zenith. Since 

\subsubsection{Neural network model}

All Hydronn retrievals are based on a common convolutional neural network
(CNN) architecture. An illustration of the architecture is provided in
Fig.~\ref{fig:architecture}. 

The posterior distribution of the retrieval is estimated by approximating the
PDF of the posterior distribution. Following \citet{metnet}, this is achieved by
discretizing the range of expected precipitation values and predicting for each
bin the probability of the observed precipitation falling it. This is
conveniently implemented by treating the retrieval as a classification problem
and using a cross-entropy loss to train the networks. During inference, the logits
predicted by the network are transformed into a probability density by applying
a softmax activation and normalizing the bin probabilities.

\subsubsection{Retrieval output}

For each retrieved pixel, the predicted posterior distribution is used to derive
the posterior mean, a sample of the posterior distribution and 14 quantiles of
the posterior distribution. The 14 quantiles are retained as a more compact
representation of the posterior distribution than the probabilities over the 128
bins that are produced by the networks. In addition to providing a direct
measure of retrieval uncertainty, the retained quantiles can be used to infer
e.g. probabilities of the instantaneous rain rates exceeding certain thresholds.

\subsubsection{Calculating precipitation accumulations}

The precipitation distributions prediced by hydronn are instantaneous
precipitaiton rates. Since GOES 16 imagery is available every 10 minutes. This
raises the question how to integrate these distributions to hourly precipitation
rates. The difficulty with this is that it is unclear how the retrieval
uncertainties should be propagated in time. For this study, we have implemented
two heuristics for doing this that represent different assumptions on the
temporal correlation of the errors in the precipitation retrieval.

The first approach is to average the predicted posterior distributions. For the
case of multiple identical observations, this approach would conserve the
retrieval uncertainties and may thus be viewed to correspond to a assuming a
strong dependence between the retrieval errors for consecutive observations.

The second approach is to assume independence of the retrieval uncertainty. For
identical, consecutive observations this will cause the retrieval uncertainty to
decay. Given the binned probability densities of two independent random
variables, the PDF of their sum can be approximated by calculating weighted
histograms of the outer sum of the bin centroids weighted by the product of the
bin probabilities.

For the evaluation of the Hydronn retrieval we calculate hourly accumulations
the PDFs for hourly accumulations using both approaches. We then calculate the
retrieval outputs described above. For each retrieval configuration two types of
hourly accumulations are produced. One corresponding to assuming dependent
retrieval error, which will we mark with '(dep.)' where necessary. And one
corresponding to assuming independent retrieval errors, which we will mark with
'(indep.)'. It should be noted, however, that this only affects the predicted
retrieval uncertainties and not the predicted mean values.


\begin{figure}
  \centering
  \includegraphics[width=0.8\textwidth]{../plots/correction_factors}
  \caption{
    A priori correction factors.
    }
  \label{fig:correction_factors}
\end{figure}

\section{Results}

\subsection{Model evaluation}

\begin{figure}
  \centering
  \includegraphics[width=1.0\textwidth]{../plots/evaluation_scatter}
  \caption{
    Scatter plots of retrieved mean precipitation against true precipitation
    for all retrieval configurations.
    }
  \label{fig:evaluation_scatter}
\end{figure}


\begin{table}
  \caption{Error metrics on test dataset. Each row displays a selection of error
     metrics for the different retrieval Hydronn configurations.}
  \label{tab:evaluation_metrics}
\begin{tabular}{l|rrrr}
  Algorithm & Bias [$\unit{mm\ h^{-1}}$] & MAE [$\unit{mm\ h^{-1}}$] & MSE [$\unit{(mm\ h^{-1})^{2}}$]& CRPS [$\ $] \\
  \hline
  \hydronnfourall & 0.0006 & 0.1626 & 1.1848 & 0.0865 \\
  \hydronntwo & -0.0001 & 0.1508 & 1.0827 & 0.0757 \\
 \end{tabular} 
\end{table}
 



\subsection{Validation against rain gauge data}

\subsubsection{Mean precipitation}
\begin{figure}
  \centering
  \includegraphics[width=1.0\textwidth]{../plots/mean_precipitation}
  \caption{
    Mean precipitation during December 2020. Each panel shows for all evaluated
    precipitation retrieval the mean precipitation in the background overlaid
    with the symmetric percentage biases with respect to gauge measurements.
    }
  \label{fig:evaluation_scatter}
\end{figure}



\begin{table}
  \caption{Error metrics for evaluation against Gauge data.}
  \label{tab:metrics}
\begin{tabular}{|l||r|r|r|r|}
\hline
  Algorithm & Bias & MSE & Correlation & SMAPE \\
\hline
\hline

PERSIANN CCS & 0.0694 & 3.9312 & 0.0915 & 1.7260 \\ 
\hline
IMERG & 0.0189 & 2.1737 & 0.3842 & 1.2665 \\ 
\hline
HYDRO & 0.1015 & 4.3813 & 0.1258 & 1.6735 \\ 
\hline
Hydronn$_{2, \text{all}}$ & 0.0128 & 1.6814 & 0.5442 & 1.0159 \\ 
\hline
Hydronn$_{2, \text{all}}$ dep. (corr.) & -0.0236 & 1.7869 & 0.5152 & 1.0135 \\ 
\hline
Hydronn$_{2, \text{all}}$ indep. (corr.) & 0.0402 & 1.7049 & 0.5448 & 1.0254 \\ 
\hline
Hydronn$_{4, \text{all}}$ & -0.0522 & 1.9128 & 0.4622 & 1.1127 \\ 
\hline
Hydronn$_{4, \text{all}}$ dep. (corr.) & -0.0459 & 1.9175 & 0.4573 & 1.0979 \\ 
\hline
Hydronn$_{4, \text{all}}$ indep. (corr.) & 0.0210 & 1.9417 & 0.4537 & 1.1927 \\ 
\hline
\end{tabular}
\end{table}

\begin{figure}
  \centering
  \includegraphics[width=1.0\textwidth]{../plots/rain_rate_distributions}
  \caption{
    Distributions of retrieved rain rates.
    }
  \label{fig:evaluation_scatter}
\end{figure}

\subsection{Uncertainty estimates}

\begin{figure}
  \centering
  \includegraphics[width=1.0\textwidth]{../plots/validation_calibration}
  \caption{
    Calibration of confidence intervals for retrieved precipitation.
    }
  \label{fig:evaluation_scatter}
\end{figure}

\subsection{Detection of high-impact events}

\begin{figure}
  \centering
  \includegraphics[width=1.0\textwidth]{../plots/validation_roc}
  \caption{
    Receiver operating characteristic (ROC) curves for the detection of precipitation
    events with a precipitation rate larger than $10\ \unit{mm\ h^{-1}}$.
    }
  \label{fig:evaluation_scatter}
\end{figure}

\begin{figure}
  \centering
  \includegraphics[width=1.0\textwidth]{../plots/validation_classification_calibration}
  \caption{
    Calibration of the probabilistic precipitation event classification. 
    }
  \label{fig:evaluation_scatter}
\end{figure}

\subsection{Case study}


\begin{figure}
  \centering
  \includegraphics[width=1.0\textwidth]{../plots/case_study}
  \caption{
    Retrieved precipitation for an extreme precipitation event near SP.
    }
  \label{fig:evaluation_scatter}
\end{figure}

\subsection{Daily cycle}


\begin{figure}
  \centering
  \includegraphics[width=1.0\textwidth]{../plots/daily_cycle}
  \caption{
    Retrieved daily cycle of precipitation.
    }
  \label{fig:evaluation_scatter}
\end{figure}
%% It is strongly recommended to make use of these sections in case data sets and/or software code have been part of your research the article is based on.

\codeavailability{TEXT} %% use this section when having only software code available


\dataavailability{TEXT} %% use this section when having only data sets available


\codedataavailability{TEXT} %% use this section when having data sets and software code available


\sampleavailability{TEXT} %% use this section when having geoscientific samples available


\videosupplement{TEXT} %% use this section when having video supplements available


\appendix
\section{}    %% Appendix A

\subsection{}     %% Appendix A1, A2, etc.


\noappendix       %% use this to mark the end of the appendix section. Otherwise the figures might be numbered incorrectly (e.g. 10 instead of 1).

%% Regarding figures and tables in appendices, the following two options are possible depending on your general handling of figures and tables in the manuscript environment:

%% Option 1: If you sorted all figures and tables into the sections of the text, please also sort the appendix figures and appendix tables into the respective appendix sections.
%% They will be correctly named automatically.

%% Option 2: If you put all figures after the reference list, please insert appendix tables and figures after the normal tables and figures.
%% To rename them correctly to A1, A2, etc., please add the following commands in front of them:

\appendixfigures  %% needs to be added in front of appendix figures

\appendixtables   %% needs to be added in front of appendix tables


\authorcontribution{TEXT} %% this section is mandatory

\competinginterests{TEXT} %% this section is mandatory even if you declare that no competing interests are present

\disclaimer{TEXT} %% optional section

\begin{acknowledgements}
TEXT
\end{acknowledgements}




%% REFERENCES



\bibliographystyle{copernicus}
\bibliography{references}
%% Since the Copernicus LaTeX package includes the BibTeX style file copernicus.bst,
%% authors experienced with BibTeX only have to include the following two lines:
%%
%% \bibliographystyle{copernicus}
%% \bibliography{example.bib}
%%
%% URLs and DOIs can be entered in your BibTeX file as:
%%
%% URL = {http://www.xyz.org/~jones/idx_g.htm}
%% DOI = {10.5194/xyz}


%% LITERATURE CITATIONS
%%
%% command                        & example result
%% \citet{jones90}|               & Jones et al. (1990)
%% \citep{jones90}|               & (Jones et al., 1990)
%% \citep{jones90,jones93}|       & (Jones et al., 1990, 1993)
%% \citep[p.~32]{jones90}|        & (Jones et al., 1990, p.~32)
%% \citep[e.g.,][]{jones90}|      & (e.g., Jones et al., 1990)
%% \citep[e.g.,][p.~32]{jones90}| & (e.g., Jones et al., 1990, p.~32)
%% \citeauthor{jones90}|          & Jones et al.
%% \citeyear{jones90}|            & 1990



%% FIGURES

%% When figures and tables are placed at the end of the MS (article in one-column style), please add \clearpage
%% between bibliography and first table and/or figure as well as between each table and/or figure.

% The figure files should be labelled correctly with Arabic numerals (e.g. fig01.jpg, fig02.png).


%% ONE-COLUMN FIGURES

%%f
%\begin{figure}[t]
%\includegraphics[width=8.3cm]{FILE NAME}
%\caption{TEXT}
%\end{figure}
%
%%% TWO-COLUMN FIGURES
%
%%f
%\begin{figure*}[t]
%\includegraphics[width=12cm]{FILE NAME}
%\caption{TEXT}
%\end{figure*}
%
%
%%% TABLES
%%%
%%% The different columns must be seperated with a & command and should
%%% end with \\ to identify the column brake.
%
%%% ONE-COLUMN TABLE
%
%%t
%\begin{table}[t]
%\caption{TEXT}
%\begin{tabular}{column = lcr}
%\tophline
%
%\middlehline
%
%\bottomhline
%\end{tabular}
%\belowtable{} % Table Footnotes
%\end{table}
%
%%% TWO-COLUMN TABLE
%
%%t
%\begin{table*}[t]
%\caption{TEXT}
%\begin{tabular}{column = lcr}
%\tophline
%
%\middlehline
%
%\bottomhline
%\end{tabular}
%\belowtable{} % Table Footnotes
%\end{table*}
%
%%% LANDSCAPE TABLE
%
%%t
%\begin{sidewaystable*}[t]
%\caption{TEXT}
%\begin{tabular}{column = lcr}
%\tophline
%
%\middlehline
%
%\bottomhline
%\end{tabular}
%\belowtable{} % Table Footnotes
%\end{sidewaystable*}
%
%
%%% MATHEMATICAL EXPRESSIONS
%
%%% All papers typeset by Copernicus Publications follow the math typesetting regulations
%%% given by the IUPAC Green Book (IUPAC: Quantities, Units and Symbols in Physical Chemistry,
%%% 2nd Edn., Blackwell Science, available at: http://old.iupac.org/publications/books/gbook/green_book_2ed.pdf, 1993).
%%%
%%% Physical quantities/variables are typeset in italic font (t for time, T for Temperature)
%%% Indices which are not defined are typeset in italic font (x, y, z, a, b, c)
%%% Items/objects which are defined are typeset in roman font (Car A, Car B)
%%% Descriptions/specifications which are defined by itself are typeset in roman font (abs, rel, ref, tot, net, ice)
%%% Abbreviations from 2 letters are typeset in roman font (RH, LAI)
%%% Vectors are identified in bold italic font using \vec{x}
%%% Matrices are identified in bold roman font
%%% Multiplication signs are typeset using the LaTeX commands \times (for vector products, grids, and exponential notations) or \cdot
%%% The character * should not be applied as mutliplication sign
%
%
%%% EQUATIONS
%
%%% Single-row equation
%
%\begin{equation}
%
%\end{equation}
%
%%% Multiline equation
%
%\begin{align}
%& 3 + 5 = 8\\
%& 3 + 5 = 8\\
%& 3 + 5 = 8
%\end{align}
%
%
%%% MATRICES
%
%\begin{matrix}
%x & y & z\\
%x & y & z\\
%x & y & z\\
%\end{matrix}
%
%
%%% ALGORITHM
%
%\begin{algorithm}
%\caption{...}
%\label{a1}
%\begin{algorithmic}
%...
%\end{algorithmic}
%\end{algorithm}
%
%
%%% CHEMICAL FORMULAS AND REACTIONS
%
%%% For formulas embedded in the text, please use \chem{}
%
%%% The reaction environment creates labels including the letter R, i.e. (R1), (R2), etc.
%
%\begin{reaction}
%%% \rightarrow should be used for normal (one-way) chemical reactions
%%% \rightleftharpoons should be used for equilibria
%%% \leftrightarrow should be used for resonance structures
%\end{reaction}
%
%
%%% PHYSICAL UNITS
%%%
%%% Please use \unit{} and apply the exponential notation


\end{document}
